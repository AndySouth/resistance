\documentclass[11pt,]{article}
\usepackage{lmodern}
\usepackage{amssymb,amsmath}
\usepackage{ifxetex,ifluatex}
\usepackage{fixltx2e} % provides \textsubscript
\ifnum 0\ifxetex 1\fi\ifluatex 1\fi=0 % if pdftex
  \usepackage[T1]{fontenc}
  \usepackage[utf8]{inputenc}
\else % if luatex or xelatex
  \ifxetex
    \usepackage{mathspec}
  \else
    \usepackage{fontspec}
  \fi
  \defaultfontfeatures{Ligatures=TeX,Scale=MatchLowercase}
\fi
% use upquote if available, for straight quotes in verbatim environments
\IfFileExists{upquote.sty}{\usepackage{upquote}}{}
% use microtype if available
\IfFileExists{microtype.sty}{%
\usepackage{microtype}
\UseMicrotypeSet[protrusion]{basicmath} % disable protrusion for tt fonts
}{}
\usepackage[margin=1in]{geometry}
\usepackage{hyperref}
\hypersetup{unicode=true,
            pdftitle={trying to get line numbers working},
            pdfauthor={Andy South and Ian Hastings},
            pdfborder={0 0 0},
            breaklinks=true}
\urlstyle{same}  % don't use monospace font for urls
\usepackage{longtable,booktabs}
\usepackage{graphicx,grffile}
\makeatletter
\def\maxwidth{\ifdim\Gin@nat@width>\linewidth\linewidth\else\Gin@nat@width\fi}
\def\maxheight{\ifdim\Gin@nat@height>\textheight\textheight\else\Gin@nat@height\fi}
\makeatother
% Scale images if necessary, so that they will not overflow the page
% margins by default, and it is still possible to overwrite the defaults
% using explicit options in \includegraphics[width, height, ...]{}
\setkeys{Gin}{width=\maxwidth,height=\maxheight,keepaspectratio}
\IfFileExists{parskip.sty}{%
\usepackage{parskip}
}{% else
\setlength{\parindent}{0pt}
\setlength{\parskip}{6pt plus 2pt minus 1pt}
}
\setlength{\emergencystretch}{3em}  % prevent overfull lines
\providecommand{\tightlist}{%
  \setlength{\itemsep}{0pt}\setlength{\parskip}{0pt}}
\setcounter{secnumdepth}{0}
% Redefines (sub)paragraphs to behave more like sections
\ifx\paragraph\undefined\else
\let\oldparagraph\paragraph
\renewcommand{\paragraph}[1]{\oldparagraph{#1}\mbox{}}
\fi
\ifx\subparagraph\undefined\else
\let\oldsubparagraph\subparagraph
\renewcommand{\subparagraph}[1]{\oldsubparagraph{#1}\mbox{}}
\fi

%%% Use protect on footnotes to avoid problems with footnotes in titles
\let\rmarkdownfootnote\footnote%
\def\footnote{\protect\rmarkdownfootnote}

%%% Change title format to be more compact
\usepackage{titling}

% Create subtitle command for use in maketitle
\newcommand{\subtitle}[1]{
  \posttitle{
    \begin{center}\large#1\end{center}
    }
}

\setlength{\droptitle}{-2em}
  \title{trying to get line numbers working}
  \pretitle{\vspace{\droptitle}\centering\huge}
  \posttitle{\par}
  \author{Andy South and Ian Hastings}
  \preauthor{\centering\large\emph}
  \postauthor{\par}
  \predate{\centering\large\emph}
  \postdate{\par}
  \date{2017-09-11}

\usepackage{setspace}
\doublespacing
\usepackage{lineno}
\linenumbers

\begin{document}
\maketitle

Seems that the header-includes bit above does work to add line numbers
and double space (although it does generate a few warning messages)

\subparagraph{Background}\label{background}

Insecticide resistance threatens the control of vectors and particularly
malaria. New insecticides are being developed to address this. Potential
strategies for using new insecticides include mixtures, sequences and
rotations and there are recommendations for how best to implement these
to manage resistance. There is, however, limited recent, accessible
modelling work addressing the evolution of resistance under different
operational strategies. Previous work concludes that preferred
strategies will be situation specific. There is potential to improve the
level of mechanistic understanding within the operational community of
how insecticide resistance can be expected to evolve in response to
different strategies.

This paper develops an accessible, mechanistic picture of how the
evolution of insecticide resistance is likely to respond to potential
intervention strategies to help guide both management and policy. The
aim is to reach an audience unlikely to read a more detailed modelling
paper. We use our model to develop a mechanistic understanding of how
insecticide resistance is expected to increase with the use of
insecticides in isolation, sequence and mixtures. The model flexibly
represents two independent genes coding for resistance to two
insecticides. We look principally at the ability of the insecticides to
kill susceptible mosquitoes, how much resistance counteracts this, the
proportion of mosquitoes that are exposed to insecticides and costs of
resistance.

\subparagraph{Results}\label{results}

We use the model to demonstrate the evolution of resistance under
different inputs and how this fits with intuitive reasoning about the
selection pressure exerted by insecticides. We show that an insecticide
used in a mixture, relative to alone, always prompts slower evolution of
resistance, but resistance to the two insecticides may evolve more
slowly when used in sequence. We show that the ability of insecticides
to kill susceptible mosquitoes (effectiveness), has the most influence
on whether resistance to two insecticides is likely to arise faster in a
mixture or sequence.

\subparagraph{Conclusions}\label{conclusions}

Our model makes more open the process of insecticide resistance
evolution and how it is likely to respond to insecticide use. We provide
a simple user-interface allowing further exploration
(\url{https://andysouth.shinyapps.io/MixSeqResist1}). These tools can
contribute to operational decisions in insecticide resistance
management.

\subsection{Keywords}\label{keywords}

insecticide resistance; public health; mosquitoes; vector-borne
diseases; infectious diseases; malaria; population genetics

\subsection{Background}\label{background-1}

Insecticide resistance is a problem for malaria {[}1{]}{[}2{]}{[}3{]}
other vector borne diseases {[}4{]} and agriculture {[}5{]}. Malaria
alone still results in hundreds of thousands of deaths per year. Recent
malaria control efforts have centred on treated bed nets and indoor
residual spraying, both reliant on insecticides. Treated nets were
recently estimated to contribute 68\% and indoor residual spraying 13\%
to averting more than 500 million falciparum malaria cases between 2000
and 2015 {[}6{]}. A recent malaria transmission model {[}7{]} predicts
that even low pyrethroid resistance is likely to increase malaria
incidence in Africa by reducing the performance of bed nets.

The WHO produced a Global Plan for Insecticide Resistance Management in
malaria vectors (GPIRM){[}1{]} which includes recommendations on
operational strategies for managing resistance including the use of
insecticide mixtures when they become available. Efforts are under way
to develop new insecticides that will be effective in the light of
existing resistance and allow additional options within insecticide
resistance management. The Innovative Vector Control Consortium (IVCC)
was set up in 2005 to develop new vector control tools and particularly
new insecticides to address insecticide resistance in malaria
transmitting mosquitoes {[}8{]}{[}9{]}. Three new insecticides are now
in development {[}9{]} and likely to be available around 2020 {[}2{]}.
It is important that decisions about how best to use the new
insecticides to delay the onset of resistance are made before
insecticides are released {[}3{]}.

Modelling studies have investigated the evolution of insecticide
resistance in insecticide mixtures including in a public health context
e.g. {[}10{]}{[}11{]}{[}12{]} but much of the work was done more than 20
years ago and there remained some confusion about the results {[}13{]}.
In an earlier paper {[}13{]} we described the technical details of a
flexible model used to investigate the relative benefits of mixtures and
sequences. Here we provide an accessible summary of the model and use
selected parameter values to describe mechanistically how the evolution
of resistance is influenced by different inputs. This mechanistic
understanding can contribute to the debate on the relative merits of
different insecticide strategies extending existing frameworks
{[}12{]}{[}4{]}{[}5{]}{[}1{]}.

My working plot.

\begin{figure}[htbp]
\centering
\includegraphics{test_linenumbers_files/figure-latex/unnamed-chunk-1-1.pdf}
\caption{Figure 1: A smiley face because I work.}
\end{figure}

\textbf{Table 2. Effect of inputs on resistance when insecticides used
singly or in sequence}

\begin{longtable}[]{@{}lll@{}}
\toprule
\begin{minipage}[b]{0.28\columnwidth}\raggedright\strut
Parameter to increase\strut
\end{minipage} & \begin{minipage}[b]{0.10\columnwidth}\raggedright\strut
effect on resistance evolution\strut
\end{minipage} & \begin{minipage}[b]{0.53\columnwidth}\raggedright\strut
Mechanism\strut
\end{minipage}\tabularnewline
\midrule
\endhead
\begin{minipage}[t]{0.28\columnwidth}\raggedright\strut
1. Effectiveness\strut
\end{minipage} & \begin{minipage}[t]{0.10\columnwidth}\raggedright\strut
faster\strut
\end{minipage} & \begin{minipage}[t]{0.53\columnwidth}\raggedright\strut
reduced fitness of SS in presence of insecticide\strut
\end{minipage}\tabularnewline
\begin{minipage}[t]{0.28\columnwidth}\raggedright\strut
2. Exposure\strut
\end{minipage} & \begin{minipage}[t]{0.10\columnwidth}\raggedright\strut
faster\strut
\end{minipage} & \begin{minipage}[t]{0.53\columnwidth}\raggedright\strut
reduced fitness of SS overall\strut
\end{minipage}\tabularnewline
\begin{minipage}[t]{0.28\columnwidth}\raggedright\strut
3. Dominance of restoration\strut
\end{minipage} & \begin{minipage}[t]{0.10\columnwidth}\raggedright\strut
faster\strut
\end{minipage} & \begin{minipage}[t]{0.53\columnwidth}\raggedright\strut
increased fitness of SR in presence of insecticide\strut
\end{minipage}\tabularnewline
\begin{minipage}[t]{0.28\columnwidth}\raggedright\strut
4. Resistance restoration\strut
\end{minipage} & \begin{minipage}[t]{0.10\columnwidth}\raggedright\strut
faster\strut
\end{minipage} & \begin{minipage}[t]{0.53\columnwidth}\raggedright\strut
increased fitness of RR in presence of insecticide\strut
\end{minipage}\tabularnewline
\begin{minipage}[t]{0.28\columnwidth}\raggedright\strut
5. Frequency\strut
\end{minipage} & \begin{minipage}[t]{0.10\columnwidth}\raggedright\strut
faster\strut
\end{minipage} & \begin{minipage}[t]{0.53\columnwidth}\raggedright\strut
less change needed to reach resistance threshold\strut
\end{minipage}\tabularnewline
\begin{minipage}[t]{0.28\columnwidth}\raggedright\strut
6. Cost of resistance\strut
\end{minipage} & \begin{minipage}[t]{0.10\columnwidth}\raggedright\strut
slower\strut
\end{minipage} & \begin{minipage}[t]{0.53\columnwidth}\raggedright\strut
reduced fitness of RR in absence of insecticide\strut
\end{minipage}\tabularnewline
\bottomrule
\end{longtable}

\textbf{Table 3. Effect of inputs on resistance when insecticides used
in a mixture}

\begin{longtable}[]{@{}lll@{}}
\toprule
\begin{minipage}[b]{0.27\columnwidth}\raggedright\strut
Parameter to increase\strut
\end{minipage} & \begin{minipage}[b]{0.12\columnwidth}\raggedright\strut
effect on resistance evolution\strut
\end{minipage} & \begin{minipage}[b]{0.52\columnwidth}\raggedright\strut
Mechanism\strut
\end{minipage}\tabularnewline
\midrule
\endhead
\begin{minipage}[t]{0.27\columnwidth}\raggedright\strut
1. Effectiveness\strut
\end{minipage} & \begin{minipage}[t]{0.12\columnwidth}\raggedright\strut
\textbf{slower}\strut
\end{minipage} & \begin{minipage}[t]{0.52\columnwidth}\raggedright\strut
one insecticide reduces the fitness of individuals resistant to the
other thereby reducing selection pressure from the other\strut
\end{minipage}\tabularnewline
\begin{minipage}[t]{0.27\columnwidth}\raggedright\strut
2. Exposure\strut
\end{minipage} & \begin{minipage}[t]{0.12\columnwidth}\raggedright\strut
faster (but less than for single)\strut
\end{minipage} & \begin{minipage}[t]{0.52\columnwidth}\raggedright\strut
reduced fitness of individuals susceptible to one insecticide increases
selection pressure for that resistance. However at the same time
selection pressure is reduced by reduced fitness of resistant
individuals caused by the other insecticide\strut
\end{minipage}\tabularnewline
\begin{minipage}[t]{0.27\columnwidth}\raggedright\strut
3. Dominance of restoration\strut
\end{minipage} & \begin{minipage}[t]{0.12\columnwidth}\raggedright\strut
faster\strut
\end{minipage} & \begin{minipage}[t]{0.52\columnwidth}\raggedright\strut
increased fitness of heterozygotes\strut
\end{minipage}\tabularnewline
\begin{minipage}[t]{0.27\columnwidth}\raggedright\strut
4. Resistance restoration\strut
\end{minipage} & \begin{minipage}[t]{0.12\columnwidth}\raggedright\strut
faster\strut
\end{minipage} & \begin{minipage}[t]{0.52\columnwidth}\raggedright\strut
increased fitness of resistants\strut
\end{minipage}\tabularnewline
\begin{minipage}[t]{0.27\columnwidth}\raggedright\strut
5. Frequency\strut
\end{minipage} & \begin{minipage}[t]{0.12\columnwidth}\raggedright\strut
faster\strut
\end{minipage} & \begin{minipage}[t]{0.52\columnwidth}\raggedright\strut
less change needed to reach resistance threshold\strut
\end{minipage}\tabularnewline
\begin{minipage}[t]{0.27\columnwidth}\raggedright\strut
6. Cost of resistance\strut
\end{minipage} & \begin{minipage}[t]{0.12\columnwidth}\raggedright\strut
slower\strut
\end{minipage} & \begin{minipage}[t]{0.52\columnwidth}\raggedright\strut
reduced fitness of resistants in absence of insecticide\strut
\end{minipage}\tabularnewline
\bottomrule
\end{longtable}

\pagebreak

\textbf{Table 4. Effect of inputs on the difference between mixture and
sequential use}

\begin{longtable}[]{@{}lll@{}}
\toprule
\begin{minipage}[b]{0.28\columnwidth}\raggedright\strut
Parameter to increase\strut
\end{minipage} & \begin{minipage}[b]{0.10\columnwidth}\raggedright\strut
increase favours mix or sequence\strut
\end{minipage} & \begin{minipage}[b]{0.53\columnwidth}\raggedright\strut
Mechanism\strut
\end{minipage}\tabularnewline
\midrule
\endhead
\begin{minipage}[t]{0.28\columnwidth}\raggedright\strut
1. Effectiveness\strut
\end{minipage} & \begin{minipage}[t]{0.10\columnwidth}\raggedright\strut
mixture\strut
\end{minipage} & \begin{minipage}[t]{0.53\columnwidth}\raggedright\strut
Higher effectiveness gives faster resistance for sequence and slower
resistance in mixture\strut
\end{minipage}\tabularnewline
\begin{minipage}[t]{0.28\columnwidth}\raggedright\strut
2. Exposure\strut
\end{minipage} & \begin{minipage}[t]{0.10\columnwidth}\raggedright\strut
sequence\strut
\end{minipage} & \begin{minipage}[t]{0.53\columnwidth}\raggedright\strut
Higher exposure gives faster resistance for sequence and mixture but the
greater effect on mixture favours sequence.\strut
\end{minipage}\tabularnewline
\begin{minipage}[t]{0.28\columnwidth}\raggedright\strut
3. Dominance of restoration\strut
\end{minipage} & \begin{minipage}[t]{0.10\columnwidth}\raggedright\strut
neither\strut
\end{minipage} & \begin{minipage}[t]{0.53\columnwidth}\raggedright\strut
Higher dominance gives faster resistance in both sequences and mixtures
such that the difference between them is not changed.\strut
\end{minipage}\tabularnewline
\begin{minipage}[t]{0.28\columnwidth}\raggedright\strut
4. Resistance restoration\strut
\end{minipage} & \begin{minipage}[t]{0.10\columnwidth}\raggedright\strut
neither\strut
\end{minipage} & \begin{minipage}[t]{0.53\columnwidth}\raggedright\strut
As for dominance of restoration. Higher resistance restoration gives
faster resistance in both sequences and mixtures such that the
difference between them is not changed.\strut
\end{minipage}\tabularnewline
\begin{minipage}[t]{0.28\columnwidth}\raggedright\strut
5. Frequency\strut
\end{minipage} & \begin{minipage}[t]{0.10\columnwidth}\raggedright\strut
neither\strut
\end{minipage} & \begin{minipage}[t]{0.53\columnwidth}\raggedright\strut
As for dominance and resistance restoration. Higher starting frequencies
give faster resistance in both sequences and mixtures such that the
difference between them is not changed.\strut
\end{minipage}\tabularnewline
\begin{minipage}[t]{0.28\columnwidth}\raggedright\strut
6. Cost of resistance\strut
\end{minipage} & \begin{minipage}[t]{0.10\columnwidth}\raggedright\strut
mixture\strut
\end{minipage} & \begin{minipage}[t]{0.53\columnwidth}\raggedright\strut
Higher costs slow the evolution of resistance more in a mixture than
when used in sequence (but with higher costs in a sequence there is a
greater chance for resistance levels to decline for the insecticide not
being used).\strut
\end{minipage}\tabularnewline
\bottomrule
\end{longtable}

\hypertarget{refs}{}
\hypertarget{ref-WHO2012}{}
1. WHO: \emph{Global plan for insecticide resistance management in
malaria vectors (GPIRM).} Geneva. 2012:130.

\hypertarget{ref-Ranson2016}{}
2. Ranson H, Lissenden N: \textbf{Insecticide Resistance in African
Anopheles Mosquitoes: A Worsening Situation that Needs Urgent Action to
Maintain Malaria Control}. \emph{Trends in Parasitology} 2016,
\textbf{32}:187--196.

\hypertarget{ref-Hemingway2016}{}
3. Hemingway J, Ranson H, Magill A, Kolaczinski J, Fornadel C, Gimnig J,
Coetzee M, Simard F, Roch DK, Hinzoumbe CK, Pickett J, Schellenberg D,
Gething P, Hoppé M, Hamon N: \textbf{Averting a malaria disaster: Will
insecticide resistance derail malaria control?} \emph{The Lancet} 2016,
\textbf{387}:1785--1788.

\hypertarget{ref-IRAC2011}{}
4. IRAC: \emph{Prevention and Management of Insecticide Resistance in
Vectors of Public Health Importance}. 2011:71.

\hypertarget{ref-FAO2012}{}
5. FAO: \emph{Guidelines on Prevention and Management of Pesticide
Resistance}. Rome; 2012:55.

\hypertarget{ref-Bhatt2015}{}
6. Bhatt S, Weiss DJ, Cameron E, Bisanzio D, Mappin B, Dalrymple U,
Battle KE, Moyes CL, Henry A, Eckhoff PA, Wenger EA, Briët O, Penny MA,
Smith TA, Bennett A, Yukich J, Eisele TP, Griffin JT, Fergus CA, Lynch
M, Lindgren F, Cohen JM, Murray CLJ, Smith DL, Hay SI, Cibulskis RE,
Gething PW: \textbf{The effect of malaria control on Plasmodium
falciparum in Africa between 2000 and 2015.} \emph{Nature} 2015,
\textbf{526}:207--11.

\hypertarget{ref-Churcher2016}{}
7. Churcher TS, Lissenden N, Griffin JT, Worrall E, Ranson H:
\textbf{The impact of pyrethroid resistance on the efficacy and
effectiveness of bednets for malaria control in Africa}. \emph{eLife}
2016, \textbf{5}:e16090.

\hypertarget{ref-Hemingway2006}{}
8. Hemingway J, Beaty BJ, Rowland M, Scott TW, Sharp BL: \textbf{The
Innovative Vector Control Consortium: improved control of mosquito-borne
diseases}. \emph{Trends in Parasitology} 2006, \textbf{22}:308--312.

\hypertarget{ref-IVCC2016}{}
9. IVCC: \textbf{Annual Report 2015-16}. 2016.

\hypertarget{ref-Curtis1985}{}
10. Curtis CF: \textbf{Theoretical models of the use of insecticide
mixtures for the management of resistance}. \emph{Bulletin of
Entomological Research} 1985, \textbf{75}:259.

\hypertarget{ref-Mani1985}{}
11. Mani GS: \textbf{Evolution of resistance in the presence of two
insecticides.} \emph{Genetics} 1985, \textbf{109}:761--783.

\hypertarget{ref-Roush1989}{}
12. Roush RT: \textbf{Designing resistance management programs: How can
you choose?} \emph{Pesticide Science} 1989, \textbf{26}:423--441.

\hypertarget{ref-Levick2017}{}
13. Levick B, South A, Hastings IM: \textbf{A Two-locus Model of The
Evolution of Insecticide Resistance to Inform and Optimise Public Health
Insecticide Deployment Strategies.} \emph{PLOS Computational Biology}
2017.


\end{document}
