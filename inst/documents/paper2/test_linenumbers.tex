\documentclass[11pt,]{article}
\usepackage{lmodern}
\usepackage{amssymb,amsmath}
\usepackage{ifxetex,ifluatex}
\usepackage{fixltx2e} % provides \textsubscript
\ifnum 0\ifxetex 1\fi\ifluatex 1\fi=0 % if pdftex
  \usepackage[T1]{fontenc}
  \usepackage[utf8]{inputenc}
\else % if luatex or xelatex
  \ifxetex
    \usepackage{mathspec}
  \else
    \usepackage{fontspec}
  \fi
  \defaultfontfeatures{Ligatures=TeX,Scale=MatchLowercase}
\fi
% use upquote if available, for straight quotes in verbatim environments
\IfFileExists{upquote.sty}{\usepackage{upquote}}{}
% use microtype if available
\IfFileExists{microtype.sty}{%
\usepackage{microtype}
\UseMicrotypeSet[protrusion]{basicmath} % disable protrusion for tt fonts
}{}
\usepackage[margin=1in]{geometry}
\usepackage{hyperref}
\hypersetup{unicode=true,
            pdftitle={trying to get line numbers working},
            pdfauthor={Andy South and Ian Hastings},
            pdfborder={0 0 0},
            breaklinks=true}
\urlstyle{same}  % don't use monospace font for urls
\usepackage{natbib}
\bibliographystyle{plainnat}
\usepackage{graphicx,grffile}
\makeatletter
\def\maxwidth{\ifdim\Gin@nat@width>\linewidth\linewidth\else\Gin@nat@width\fi}
\def\maxheight{\ifdim\Gin@nat@height>\textheight\textheight\else\Gin@nat@height\fi}
\makeatother
% Scale images if necessary, so that they will not overflow the page
% margins by default, and it is still possible to overwrite the defaults
% using explicit options in \includegraphics[width, height, ...]{}
\setkeys{Gin}{width=\maxwidth,height=\maxheight,keepaspectratio}
\IfFileExists{parskip.sty}{%
\usepackage{parskip}
}{% else
\setlength{\parindent}{0pt}
\setlength{\parskip}{6pt plus 2pt minus 1pt}
}
\setlength{\emergencystretch}{3em}  % prevent overfull lines
\providecommand{\tightlist}{%
  \setlength{\itemsep}{0pt}\setlength{\parskip}{0pt}}
\setcounter{secnumdepth}{0}
% Redefines (sub)paragraphs to behave more like sections
\ifx\paragraph\undefined\else
\let\oldparagraph\paragraph
\renewcommand{\paragraph}[1]{\oldparagraph{#1}\mbox{}}
\fi
\ifx\subparagraph\undefined\else
\let\oldsubparagraph\subparagraph
\renewcommand{\subparagraph}[1]{\oldsubparagraph{#1}\mbox{}}
\fi

%%% Use protect on footnotes to avoid problems with footnotes in titles
\let\rmarkdownfootnote\footnote%
\def\footnote{\protect\rmarkdownfootnote}

%%% Change title format to be more compact
\usepackage{titling}

% Create subtitle command for use in maketitle
\newcommand{\subtitle}[1]{
  \posttitle{
    \begin{center}\large#1\end{center}
    }
}

\setlength{\droptitle}{-2em}
  \title{trying to get line numbers working}
  \pretitle{\vspace{\droptitle}\centering\huge}
  \posttitle{\par}
  \author{Andy South and Ian Hastings}
  \preauthor{\centering\large\emph}
  \postauthor{\par}
  \predate{\centering\large\emph}
  \postdate{\par}
  \date{2017-04-11}

\usepackage{setspace}
\doublespacing
\usepackage{lineno}
\linenumbers

\begin{document}
\maketitle

\subparagraph{Background}\label{background}

Insecticide resistance threatens the control of vectors and particularly
malaria. New insecticides are being developed to address this. Potential
strategies for using new insecticides include mixtures, sequences and
rotations and there are recommendations for how best to implement these
to manage resistance. There is, however, limited recent, accessible
modelling work addressing the evolution of resistance under different
operational strategies. Previous work concludes that preferred
strategies will be situation specific. There is potential to improve the
level of mechanistic understanding within the operational community of
how insecticide resistance can be expected to evolve in response to
different strategies.

This paper develops an accessible, mechanistic picture of how the
evolution of insecticide resistance is likely to respond to potential
intervention strategies to help guide both management and policy. The
aim is to reach an audience unlikely to read a more detailed modelling
paper. We use our model to develop a mechanistic understanding of how
insecticide resistance is expected to increase with the use of
insecticides in isolation, sequence and mixtures. The model flexibly
represents two independent genes coding for resistance to two
insecticides. We look principally at the ability of the insecticides to
kill susceptible mosquitoes, how much resistance counteracts this, the
proportion of mosquitoes that are exposed to insecticides and costs of
resistance.

\subparagraph{Results}\label{results}

We use the model to demonstrate the evolution of resistance under
different inputs and how this fits with intuitive reasoning about the
selection pressure exerted by insecticides. We show that an insecticide
used in a mixture, relative to alone, always prompts slower evolution of
resistance, but resistance to the two insecticides may evolve more
slowly when used in sequence. We show that the ability of insecticides
to kill susceptible mosquitoes (effectiveness), has the most influence
on whether resistance to two insecticides is likely to arise faster in a
mixture or sequence.

\subparagraph{Conclusions}\label{conclusions}

Our model makes more open the process of insecticide resistance
evolution and how it is likely to respond to insecticide use. We provide
a simple user-interface allowing further exploration
(\url{https://andysouth.shinyapps.io/MixSeqResist1}). These tools can
contribute to operational decisions in insecticide resistance
management.

\subsection{Keywords}\label{keywords}

insecticide resistance; public health; mosquitoes; vector-borne
diseases; infectious diseases; malaria; population genetics

\subsection{Background}\label{background-1}

Insecticide resistance is a problem for malaria
\citep{WHO2012}\citep{Ranson2016}\citep{Hemingway2016} other vector
borne diseases \citep{IRAC2011} and agriculture \citep{FAO2012}. Malaria
alone still results in hundreds of thousands of deaths per year. Recent
malaria control efforts have centred on treated bed nets and indoor
residual spraying, both reliant on insecticides. Treated nets were
recently estimated to contribute 68\% and indoor residual spraying 13\%
to averting more than 500 million falciparum malaria cases between 2000
and 2015 \citep{Bhatt2015}. A recent malaria transmission model
\citep{Churcher2016} predicts that even low pyrethroid resistance is
likely to increase malaria incidence in Africa by reducing the
performance of bed nets.

The WHO produced a Global Plan for Insecticide Resistance Management in
malaria vectors (GPIRM)\citep{WHO2012} which includes recommendations on
operational strategies for managing resistance including the use of
insecticide mixtures when they become available. Efforts are under way
to develop new insecticides that will be effective in the light of
existing resistance and allow additional options within insecticide
resistance management. The Innovative Vector Control Consortium (IVCC)
was set up in 2005 to develop new vector control tools and particularly
new insecticides to address insecticide resistance in malaria
transmitting mosquitoes \citep{Hemingway2006}\citep{IVCC2016}. Three new
insecticides are now in development \citep{IVCC2016} and likely to be
available around 2020 \citep{Ranson2016}. It is important that decisions
about how best to use the new insecticides to delay the onset of
resistance are made before insecticides are released
\citep{Hemingway2016}.

Modelling studies have investigated the evolution of insecticide
resistance in insecticide mixtures including in a public health context
e.g. \citep{Curtis1985}\citep{Mani1985}\citep{Roush1989} but much of the
work was done more than 20 years ago and there remained some confusion
about the results \citep{Levick2017}. In an earlier paper
\citep{Levick2017} we described the technical details of a flexible
model used to investigate the relative benefits of mixtures and
sequences. Here we provide an accessible summary of the model and use
selected parameter values to describe mechanistically how the evolution
of resistance is influenced by different inputs. This mechanistic
understanding can contribute to the debate on the relative merits of
different insecticide strategies extending existing frameworks
\citep{Roush1989}\citep{IRAC2011}\citep{FAO2012}\citep{WHO2012}.

My working plot.

\begin{figure}[htbp]
\centering
\includegraphics{test_linenumbers_files/figure-latex/unnamed-chunk-1-1.pdf}
\caption{Figure 1: A smiley face because I work.}
\end{figure}

\bibliography{library}


\end{document}
